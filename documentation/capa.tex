
% http://bcc.ufabc.edu.br/documentos/normalizacao.pdf

\renewcommand{\nomes}
{
	\begin{table}[H]
		\centering
		\begin{tabular}{c}

			Heitor Rodrigues Savegnago \\

		\end{tabular}
	\end{table}
}

\renewcommand{\centro}{Centro de Matemática, Computação e Cognição}
\renewcommand{\centroSigla}{CMCC}
% \renewcommand{\centro}{Centro de Engenharia, Modelagem e Ciências Sociais Aplicadas}
% \renewcommand{\centroSigla}{CECS}

% \renewcommand{\disciplina}{Trabalho de Graduação I em Engenharia de Informação}
% \renewcommand{\codigoDisciplina}{ESTI902-17}

% \renewcommand{\disciplina}{Projeto de Graduação em Computação I}
% \renewcommand{\codigoDisciplina}{MCTA029-17}

% \renewcommand{\grupo}{Grupo 1}

\renewcommand{\titulo}{Identificação de medicamentos utilizando técnicas de visão computacional}

\renewcommand{\professor}{Prof. Dr. Francisco de Assis Zampirolli}
% \renewcommand{\professor}{Prof. Dr. Ivan Roberto de Santana Casella}

\renewcommand{\local}{Santo André, SP}

\renewcommand{\data}{2024}

\renewcommand{\notaDeRosto}
{
	Trabalho de Conclusão de Curso apresentado ao \centro{} da Universidade Federal do ABC como requisito parcial à obtenção do título de Bacharel em Ciência da Computação.

    \vspace{1em}

    Orientador: \professor.
}

\renewcommand{\agradecimentos}
{

	Agradeço aos meus pais, que sempre me incentivaram e deram suporte pra seguir atrás dos meus sonhos;

	Ao meu orientador do presente trabalho, Dr. Francisco de Assis Zampirolli, por aceitar me orientar e acompanhar pacientemente ao longo do desenvolvimento deste projeto;

	Ao meu orientador do Trabalho de Graduação em Engenharia de Informação, Dr. Ivan Roberto de Santana Casella, que pacientemente seguiu me orientando mesmo eu levando mais tempo que o esperado;

	Aos amigos, aos colegas e aos familiares que me mandaram tantas das fotos de remédios que compuzeram o banco usado;

	Ao meu irmão, por me ajudar a entender os problemas que tive programando, mesmo que só estivesse lá pra me ouvir falar a respeito.

	Aos amigos que tantas vezes pedi opiniões sobre como escrever e descrever tantas das coises neste trabalho;

	À Universidade Federal do ABC, seus docentes, técnicos e terceirizados, sem o qual eu não poderia chegar aqui;

	E a todos que contribuíram, direta ou indiretamente, no meu caminho até aqui, meu muito obrigado!

}
