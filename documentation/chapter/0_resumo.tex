A necessidade de medicamentos de uso contínuo tem se tornando cada vez mais comum, especialmente com o aumento da longevidade da população.
Junto da idade avançada também se torna mais comum a presença de problemas relacionados à visão, principalmente a presbiopia, que pode causar dificuldade de leitura.
Estes fatores juntos podem trazer um novo problema, onde pacientes confundes embalagens de medicamentos por serem visualmente semelhantes.
Este trabalho propõe um sistema capaz de identificar e trazer informações sobre um medicamento, através dos elementos textuais da embalagem, utilizando visão computacional.
Com este propósito, foi construído, em Python, um sistema que recebe um arquivo de imagem e realiza a leitura do texto, utilizando diferentes codificações de cores para esta imagem, método que se mostrou eficaz para a identificação dos termos na imagem.
Se forem encontrados termos, o sistema os classifica e busca no Bulário Eletrônico da Anvisa e, em caso de resposta válida, carrega o arquivo da bula.
Para testar este sistema, foi construído um banco de imagens com fotos de medicamentos, com cerca de \num{1200} imagens.
Ao longo do desenvolvimento, foram notados problemas de acesso \textit{online} ao portal do Bulário Eletrônico, que fica inacessível em certos horários do dia.
Também foram notados problemas relacionados à orientação do texto nas imagens, onde se fez necessário ajustar algumas imagens para a correta leitura.
A acurácia do sistema para leitura total ou parcial dos termos de interesse foi de pouco mais de \SI{80}{\percent} do banco de imagens, e destes termos, obteve sucesso total ou parcial ao encontrar o arquivo em mais de \SI{85}{\percent} dos casos.
Estes resultados se mostraram melhores que os cerca \SI{26}{\percent} de leitura parcial ou correta realizadas pelo sistema semelhante que não utiliza de diferentes codificações de cores, que mostra o ganho envolvido na performance através deste método.

\paragraph*{Palavras-chave:} Anvisa; Bulário Eletrônico; Tecnologia Assistiva; Processamento Digital de Imagens; Visão Computacional; Reconhecimento Ótico de Caracteres; Python.
