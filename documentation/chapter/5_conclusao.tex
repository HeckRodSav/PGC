\chapter{Conclusão}\label{cap:conclusao}

Medicamentos de uso contínuo podem proporcionar bem estar a pacientes com problemas crônicos de saúde.
Porém devem ser ministrados com cuidado, já que embalagens podem ser confundidas.
Também é relevante destacar o risco relacionado ao hábito de se automedicar, comum no povo brasileiro, o que pode disfarçar sintomas de doenças graves, e retardar seu diagnóstico.

Durante a revisão bibliográfica, foram listadas algumas propostas com escopo semelhante, focando em auxiliar pessoas com problemas de visão ou idade avançada, e também relacionadas a identifcação de medicamentos, que mostra a relevância do assunto.

Como o Bulário Eletrônico da \ac{Anvisa} não possui uma \ac{API}, foi necessário realizar o acesso aos dados pelo portal.
Apesar de ser um método funcional, não conta com ferramentas de confirmação de estado do servidor de dados, então se este estiver inoperante, não é retornado um aviso, apenas um erro de acesso.
Associado a essa limitação, houve o problema de acesso ao portal da \ac{Anvisa} fora do horário comercial, que atrasou as análises em algumas situações ao longo do desenvolvimento do projeto.

As análises iniciais, relacionadas à performance do motor \ac{OCR} utilizado, apontaram que o algoritmo não seria suficiente para encontrar o texto de interesse em toda imagem.
A adição de versões diferentes da imgem, operando em outras codificações de cores, acrescentou maior amplitude ao algoritmo utilizado, porém deixou o tempo de processamento maior.

Normalizar os termos encontrados facilitou a busca destes no Bulário Eletrônico, porém, em alguns casos, o registro do medicamente contém caracteres especiais, isto é, que fogem da codificação \ac{ASCII}.
Para estes, foi necessário adicionar à lista de termos versões corrigidas das palavras.

A ordenação dos termos encontrados pelo motor \ac{OCR} foi uma das principais dificuldades encontradas, já que várias embalagens apresentam os compostos do princípio ativo associado, e estes compostos podem nomear outros medicamentos.
Utilizar das características geométricas do termo na imagem foi uma forma encotrada para contornar isso, mas não foi suficiente para resolver todos os casos.
Nessa situação, mesmo que os termos apropriados tenham sido localizados, esses poderiam ser buscados depois de outros, que retornariam um medicamento diferente do correto.

Buscar um termo localizado gera uma lista de resultados, em muitos casos vazia, quando o termo não está presente em qualquer nome de medicamento, em outros casos com valores relevantes, quando o termo realmente faz parte do nome de um medicamento.
Porém houveram casos notáveis, onde a busca resulta em uma lista com centenas de medicamentos, quando o termo buscado é composto somente por um caractere, geralmente a lista é composta por todos os medicamentos que o nome começa com este caractere.
Para contornar isso, os termos com um único caractere não são buscados, porém estes termos ainda são mantidos na lista, já que podem representar a principal diferença entre o nome de medicamentos, \eg\ diferenciar nomes como ``Vitamina C'' de ``Vitamina D''.

Os resultados mostraram que, apesar dos problemas encontrados, o sistema desenvolvido realizou a identificação correta ou parcial do termo de interesse em mais de \SI{80}{\percent} do banco de fotos.
Dos termos lidos correta ou parcialmente, conseguiu localizar mais de \SI{85}{\percent} no sistema do Bulário Eletrônico da \ac{Anvisa}.
Apesar de identificados corretamente, algumas imagens do banco não foram localizadas, estas podem ser agrupadas em dois casos principais, o primeiro onde a foto se refere a algo que não está listado no Bulário Eletrônico da Anvisa, como medicamentos homeopáticos ou para animais, e o segundo para casos onde a ordem que os termos foram buscados resultou num resultado errôneo.

Foi analisada também uma versão do sistema onde somente a imagem original é utilizada na busca de imagens, \ie\ sem as análises em diferentes sistemas de codificação de cores.
Neste caso, pouco mais de \SI{25}{\percent} das imagens tiverem seus termos de interesse identificados correta ou parcialmente.
Destes termos, mais de \SI{95}{\percent} foram localizados no Bulário Eletrônico da \ac{Anvisa}.
É importante notar que, apesar de ser uma porcentagem relativamente alta, estes termos encontrados são referentes a menos de um quarto do banco de imagens.
Essa análise alternativa mostra que o método adotado foi relevante na localização dos termos de interesse e, consequentemente, na correta localização no Bulário Eletrônico da \ac{Anvisa}.

A limitação de serviço no portal do Bulário Eletrônico da \ac{Anvisa} refletiu em uma limitação equivalente no sistema aqui desenvolvido.
Como o portal fica indisponível fora do horário comercial, não é possível realizar buscas neste período, mesmo que o sistema tenha encontrado e classificado os termos corretamente.

Outros problemas encontrados foram relacionados à forma que os termos de interesse estavam dispostos nas imagens.
Em alguns casos, a angulação do texto impediu o correto funcionamento do motor \ac{OCR}, em outros, reflexos e obstruções impediram a identificação.
Algumas dessas obstruções consistem em fissuras associadas a própria abertura da embalagem do medicamento, como cartelas de comprimidos.

Em conclusão, o trabalho aqui proposto se mostrou eficaz na identificação de medicamentos e busca dos arquivos de bulas eletrônicas registrados na \ac{Anvisa}, obtendo sucesso ao localizar mais de \SI{70}{\percent} dos casos testados.
A abordagem, utilizando diferentes codificações de cores, se mostrou valorosa para os resultados, apesar de aumentar o tempo de processamento das imagens.

Em trabalhos futuros, há a possibilidade de melhorar o método para busca dos termos encontrados no Bulário Eletrônico da \ac{Anvisa}, tornando-a mais criteriosa em relação aos resultados, diminuindo a incidência de falsos positivos.
É possível realizar uma pré indexação \textit{offline} dos medicamentos presentes no Bulário Eletrônico, viabilizando grande melhoria nos métodos de busca, que a tornaria mais rápida e eficiente, além de garantir o funcionamento em qualquer horário.
Além disso, o desenvolvimento de uma interface de usuário mais amigável, com uma versão \textit{mobile}, poderia tornar o sistema mais acessível, além de poder contar com avisos sobre o risco da automedicação.
Também é viável buscar métodos para lidar melhor com a angulação do texto nas imagens, simplificando o pré processamento do banco de fotos.
Outra possível melhoria consiste no levantamento dos medicamentos que o usuário mais utilizou recentemente, viabilizando melhor identificação para casos de fotos com problemas visuais.

% \MexerDepois{Completar isso depois}
