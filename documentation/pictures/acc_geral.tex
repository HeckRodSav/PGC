\begin{tikzpicture}
    % \pgfsetfillopacity{0.5}

    \begin{axis} %configuração do eixo Y esquerdo e eixo X
    [
        % reverse legend, % inverte a ordem que os items aparecem na legenda
    	legend style={
			% at={($(current bounding box.north)$)},
			at={($(current bounding box.north)+(1cm,0cm)$)},
        	anchor=south,
        	legend columns=3,
        % 	transpose legend,
        	draw=none,
			/tikz/every even column/.append style={column sep=0.5cm}
    	}, % onde exibir
        % axis x line=center,
        % axis y line=center,
        height=.4\textwidth, % altura da região do gráfico
        width=0.8\textwidth, % largura da região do gráfico
        scale only axis, %
        minor grid style={densely dotted}, % estilo da grade secundária
        major grid style={densely dashed}, % estilo da grade principal
        grid style={lightgray}, % cor das grades
        % axis on top, % forçar grade para ficar por cima do gráfico
        %
        %
        % axis y line*=left, % define gráfico para usar eixo esquerdo sem exibir direito
        ylabel={Leitura}, % titulo eixo vertical
        % y tick label style={
        %     /pgf/number format/.cd,
        %     fixed,
        %     % fixed zerofill,
        %     % precision=0, % quantidade de casas depois da virgula
        %     /tikz/.cd
        % },
        % y filter/.expression={y==0 ? NaN : y},
        % scaled y ticks = false,
        % ymode=log,
        % yticklabel={\pgfmathparse{\tick/10^3}\pgfmathprintnumber{\pgfmathresult}}, % fator multiplicativo para valores do eixo
        symbolic y coords={ Correta, Parcial, Incorreta},
        ytick=data,
        % y tick label style={/pgf/number format/1000 sep=}, % Altera marcação de milhar
        y tick label style={rotate=90},
        % yticklabel style={rotate=90},
        %ytick={0,100,200,300, 400, 500, 600, 700, 800}, % lista de valores a serem utilizados no eixo
        % ymin=-5,  ymax=85,  % intervalo de valores no eixo y -> na dúvida, deixe comentado
        %
        % ymajorgrids=true, % exibir grade principal y
        % yminorgrids=true, % exibir grade secundária y
        % minor y tick num=3, % contagem de linhas na grade secundária y
        % ybar,
        %
        %
        xlabel={Casos analisados (\si{\percent})}, % título eixo horizontal
        % xticklabel={\pgfmathparse{\tick*10^3}\pgfmathprintnumber{\pgfmathresult}}, % fator multiplicativo para valores do eixo
        % xmode=log,
        % log ticks with fixed point,
        % x filter/.code=\pgfmathparse{#1 + 6.90775527898214},
        x tick label style={
            /pgf/number format/.cd,
            fixed,
            % fixed zerofill,
            precision=0,
            /tikz/.cd,
            /pgf/number format/use comma
        },
		% symbolic x coords={Nao encontrado, Semelhante, Encontrado},
		% xtick=data,
		% x tick label style={rotate=30,anchor=east},
        xmin=-5,
        xmax=85, % intervalo de valores no eixo x -> na dúvida, deixe comentado
        % scaled x ticks = true,
        %
        xmajorgrids=true, % exibir grade principal x
        xminorgrids=true, % exibir grade secundária x
        minor x tick num=3, % contagem de linhas na grade secundária x
        %
        %
        %
        enlarge y limits=0.25,
        % enlargelimits=0.5,
		xbar stacked,
		bar width = 1.5cm
    ]

    % \addplot[mark=none,red]
    % table[
    %     x=time, % cabeçalho da coluna de dados X no arquivo
    %     y=vin % cabeçalho da coluna de dados Y no arquivo
    % ]
    % {graficos/dados/C.2.2.dat};
    % \addlegendentry{V\textsubscript{in}}

	\addplot[mark=none, draw opacity=0, fill=cmyk_R]
	table[
		y=itens, % cabeçalho da coluna de dados X no arquivo
		x={Não encontrado} % cabeçalho da coluna de dados Y no arquivo
	]
	{../data/acc_geral.dat};
	\addlegendentry{Não encontrado}

	\addplot[mark=none, draw opacity=0, fill=cmyk_G]
	table[
		y=itens, % cabeçalho da coluna de dados X no arquivo
		x=Semelhante % cabeçalho da coluna de dados Y no arquivo
	]
	{../data/acc_geral.dat};
	\addlegendentry{Semelhante}

	\addplot[mark=none, draw opacity=0, fill=cmyk_B]
    table[
        y=itens, % cabeçalho da coluna de dados X no arquivo
        x=Encontrado % cabeçalho da coluna de dados Y no arquivo
    ]
    {../data/acc_geral.dat};
    \addlegendentry{Encontrado}

    \end{axis}
\end{tikzpicture}
